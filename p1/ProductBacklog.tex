\documentclass[a4paper, 12pt, spanish]{memoria}

\usepackage{ings}
\ConfigurarDocumento{
    practica = {1}{Práctica 1: Product Backlog},
}

\begin{document}

\section*{Product Backlog}

%% Formatos validos: list, tikz, table
\ProductBacklog[format=table]{%
{%
    identificador = S01.01,
    enunciado     = {Como director de proyecto, ingeniero o cliente quiero autenticarme en el sistema para acceder a mis opciones y mis proyectos.},
    valor cliente = 5,
    esfuerzo      = {2}{3},
    iteracion     = 1,
    prioridad     = 5,
    comentarios   = {Funcionalidad básica de la aplicación.},
},{%
    identificador = S01.02,
    enunciado     = {Como director de proyecto quiero poder crear proyectos para añadir nuevos proyectos },
    valor cliente = 5,
    esfuerzo      = {1}{1},
    iteracion     = 1,
    prioridad     = 5,
    comentarios   = {Funcionalidad básica de la aplicación.},
},{%
    identificador = S01.03,
    enunciado     = {Como director de proyecto, ingeniero o cliente quiero listar los proyectos en los que estoy involucrado para poder acceder a ellos.},
    valor cliente = 4,
    esfuerzo      = {1}{1},
    iteracion     = 1,
    prioridad     = 4,
    comentarios   = {},
},{%
    identificador = S01.04,
    enunciado     = {Como director de proyecto o ingeniero quiero acceder a los requisitos y solicitudes asociadas a un proyecto para poder visualizarlos y comprobar si son correctos y completos.},
    valor cliente = 5,
    esfuerzo      = {3}{2},
    iteracion     = 1,
    prioridad     = 4,
    comentarios   = {},
},{%
    identificador = S01.05,
    enunciado     = {Como director de proyecto o ingeniero quiero poder añadir requisitos a mis proyectos siguiendo un formato estándar y desde la propia aplicación para unificar todos los requisitos.},
    valor cliente = 5,
    esfuerzo      = {2}{2},
    iteracion     = 1,
    prioridad     = 5,
    comentarios   = {},
},{%
    identificador = S01.06,
    enunciado     = {Como usuario quiero modificar requisitos de mis proyectos para completarlos o corregirlos en etapas posteriores a la creación.},
    valor cliente = 2,
    esfuerzo      = {2}{2},
    iteracion     = 2,
    prioridad     = 2,
    comentarios   = {},
},{%
    identificador = S01.07,
    enunciado     = {Como usuario quiero consultar los requisitos de mis proyectos en función de ciertos parámetros para poder encontrarlos más fácilmente.},
    valor cliente = 1,
    esfuerzo      = {1}{1},
    iteracion     = 2,
    prioridad     = 1,
    comentarios   = {},
},{%
    identificador = S01.08,
    enunciado     = {Como usuario quiero crear solicitudes una vez se haya hecho la primera etapa de deducción de requisitos para poder debatir la modificación o inclusión de nuevos requisitos en el proyecto.},
    valor cliente = 3,
    esfuerzo      = {2}{2},
    iteracion     = 2,
    prioridad     = 3,
    comentarios   = {},
},{%
    identificador = S01.09,
    enunciado     = {Como ingeniero quiero crear artefactos para poder añadir prototipos, diagramas de transición de estados, componentes, clases, planes de prueba, casos de prueba y otros para el desarrollo de los requisitos.},
    valor cliente = 5,
    esfuerzo      = {4}{3},
    iteracion     = 1,
    prioridad     = 5,
    comentarios   = {},
},{%
    identificador = S01.10,
    enunciado     = {Como ingeniero quiero asociar artefactos creados a requisitos de un proyecto para tenerlos agrupados en el requisito correspondiente.},
    valor cliente = 4,
    esfuerzo      = {1}{1},
    iteracion     = 1,
    prioridad     = 4,
    comentarios   = {},
},{%
    identificador = S01.11,
    enunciado     = {Como director de proyecto o ingeniero quiero listar los artefactos de un proyecto para consultarlos agrupados por requisitos.},
    valor cliente = 3,
    esfuerzo      = {1}{1},
    iteracion     = 1,
    prioridad     = 3,
    comentarios   = {},
},{%
    identificador = S01.12,
    enunciado     = {Como director de proyecto quiero crear un reporte básico con información de los requisitos del proyecto y su estado para mostrárselo al cliente y ver si la planificación se estará llevando a cabo.},
    valor cliente = 5,
    esfuerzo      = {4}{3},
    iteracion     = 2,
    prioridad     = 4,
    comentarios   = {},
},{%
    identificador = S01.13,
    enunciado     = {Como director de proyecto quiero asignar horarios a los ingenieros controlando el número de horas para hacer un reparto de tareas sin sobreasignaciones de trabajo.},
    valor cliente = 3,
    esfuerzo      = {3}{3},
    iteracion     = 2,
    prioridad     = 2,
    comentarios   = {},
}} % /ProductBacklog

\end{document}