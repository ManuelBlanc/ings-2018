\documentclass[a4paper, 12pt, spanish]{memoria}

\usepackage{ings}
\ConfigurarDocumento{
    practica = {1}{Práctica 1: Product Backlog},
}

\begin{document}

\section*{Product Backlog}

%% Formatos validos: none, list, tikz, table
\ProductBacklog[format=tikz]{%
{%
    identificador = Autenticacion,
    enunciado     = {%
        como = {director de proyecto, ingeniero o cliente},
        quiero = {autenticarme en el sistema},
        para = {acceder a mis opciones y mis proyectos},
    },
    valor cliente = 5,
    esfuerzo      = {2}{3},
    iteracion     = 1,
    prioridad     = 5,
    comentarios   = {Funcionalidad básica de la aplicación},
},{%
    identificador = CrearProyectos,
    enunciado     = {%
        como = {director de proyecto},
        quiero = {poder crear proyectos},
        para = {añadir nuevos proyectos },
    },
    valor cliente = 5,
    esfuerzo      = {1}{1},
    iteracion     = 1,
    prioridad     = 5,
    comentarios   = {Funcionalidad básica de la aplicación},
},{%
    identificador = ListarProyectos,
    enunciado     = {%
        como = {director de proyecto, ingeniero o cliente},
        quiero = {listar los proyectos en los que estoy involucrado},
        para = {poder acceder a ellos},
    },
    valor cliente = 4,
    esfuerzo      = {1}{1},
    iteracion     = 1,
    prioridad     = 4,
    comentarios   = {},
},{%
    identificador = ListarRequisitos,
    enunciado     = {%
        como = {director de proyecto o ingeniero},
        quiero = {acceder a los requisitos asociadas a un proyecto},
        para = {poder visualizarlos y comprobar si son correctos y completos},
    },
    valor cliente = 5,
    esfuerzo      = {3}{2},
    iteracion     = 1,
    prioridad     = 4,
    comentarios   = {},
},{%
    identificador = ListarSolicitudes,
    enunciado     = {%
        como = {director de proyecto o ingeniero},
        quiero = {acceder a las solicitudes asociadas a un proyecto},
        para = {poder visualizarlos y comprobar si son correctos y completos},
    },
    valor cliente = 5,
    esfuerzo      = {3}{2},
    iteracion     = 1,
    prioridad     = 4,
    comentarios   = {},
},{%
    identificador = CrearRequisitos,
    enunciado     = {%
        como = {director de proyecto o ingeniero},
        quiero = {poder añadir requisitos a mis proyectos siguiendo un formato estándar y desde la propia aplicación},
        para = {unificar todos los requisitos},
    },
    valor cliente = 5,
    esfuerzo      = {2}{2},
    iteracion     = 1,
    prioridad     = 5,
    comentarios   = {},
},{%
    identificador = ModificarRequisitos,
    enunciado     = {%
        como = {usuario},
        quiero = {modificar requisitos de mis proyectos},
        para = {completarlos o corregirlos en etapas posteriores a la creación},
    },
    valor cliente = 2,
    esfuerzo      = {2}{2},
    iteracion     = 2,
    prioridad     = 2,
    comentarios   = {},
},{%
    identificador = FiltrarRequisitos,
    enunciado     = {%
        como = {usuario},
        quiero = {consultar los requisitos de mis proyectos en función de ciertos parámetros},
        para = {poder encontrarlos más fácilmente},
    },
    valor cliente = 1,
    esfuerzo      = {1}{1},
    iteracion     = 2,
    prioridad     = 1,
    comentarios   = {},
},{%
    identificador = CrearSolicitudes,
    enunciado     = {%
        como = {usuario},
        quiero = {crear solicitudes una vez se haya hecho la primera etapa de deducción de requisitos},
        para = {poder debatir la modificación o inclusión de nuevos requisitos en el proyecto},
    },
    valor cliente = 3,
    esfuerzo      = {2}{2},
    iteracion     = 2,
    prioridad     = 3,
    comentarios   = {},
},{%
    identificador = CrearArtefactos,
    enunciado     = {%
        como = {ingeniero},
        quiero = {crear artefactos para poder añadir prototipos, diagramas de transición de estados, componentes, clases, planes de prueba, casos de prueba y otros},
        para = {el desarrollo de los requisitos},
    },
    valor cliente = 5,
    esfuerzo      = {4}{3},
    iteracion     = 1,
    prioridad     = 5,
    comentarios   = {},
},{%
    identificador = AsignarArtefactos,
    enunciado     = {%
        como = {ingeniero},
        quiero = {asociar artefactos creados a requisitos de un proyecto},
        para = {tenerlos agrupados en el requisito correspondiente},
    },
    valor cliente = 4,
    esfuerzo      = {1}{1},
    iteracion     = 1,
    prioridad     = 4,
    comentarios   = {},
},{%
    identificador = ListarArtefactos,
    enunciado     = {%
        como = {director de proyecto o ingeniero},
        quiero = {listar los artefactos de un proyecto},
        para = {consultarlos agrupados por requisitos},
    },
    valor cliente = 3,
    esfuerzo      = {1}{1},
    iteracion     = 1,
    prioridad     = 3,
    comentarios   = {},
},{%
    identificador = CrearReporte,
    enunciado     = {%
        como = {director de proyecto},
        quiero = {crear un reporte básico con información de los requisitos del proyecto y su estado},
        para = {mostrárselo al cliente y ver si la planificación se estará llevando a cabo},
    },
    valor cliente = 5,
    esfuerzo      = {4}{3},
    iteracion     = 2,
    prioridad     = 4,
    comentarios   = {},
},{%
    identificador = AsignarHorarios,
    enunciado     = {%
        como = {director de proyecto},
        quiero = {asignar horarios a los ingenieros controlando el número de horas},
        para = {hacer un reparto de tareas sin sobreasignaciones de trabajo},
    },
    valor cliente = 3,
    esfuerzo      = {3}{3},
    iteracion     = 2,
    prioridad     = 2,
    comentarios   = {},
},{%
    identificador = S01.14,
    enunciado     = {Como jefe de proyecto quiero poder aceptar o rechazar solicitudes  de inclusión de requisitos para incluir o rechazar las tareas solicitadas por el cliente.},
    valor cliente = 3,
    esfuerzo      = {2}{3},
    iteracion     = 2,
    prioridad     = 1,
    comentarios   = {},
},{%
    identificador = S01.15,
    enunciado     = {Como jefe de proyecto quiero añadir requisitos a mis proyectos para posteriormente asignar tareas a los ingenieros},
    valor cliente = 3,
    esfuerzo      = {2}{2},
    iteracion     = 2,
    prioridad     = 3,
    comentarios   = {},
}} % /ProductBacklog

Me mola \ref{CrearRequisitos}

\end{document}