\documentclass[a4paper, 12pt, spanish]{memoria}

\usepackage{ings}
\AtBeginDocument{\section*{Product Backlog}}

\newcounter{reqId}
\pgfqkeys{/ings/backlog}{
  /deck/card/add key/.list = {%
    identificador, valor cliente,
    iteracion, prioridad, comentarios
  },
  enunciado/.code = {%
    \pgfqkeys{/ings/backlog/enunciado}{%
      como/.initial = {COMO},
      quiero/.initial = {QUIERO},
      para/.initial = {PARA},
      #1,
    }%
    \pgfqkeys{/ings/backlog}{%
      enunciado/.initial = {%
        \textsc{como} \pgfkeysvalueof{/ings/backlog/enunciado/como} \newline
        \textsc{quiero} \pgfkeysvalueof{/ings/backlog/enunciado/quiero} \newline
        \textsc{para} \pgfkeysvalueof{/ings/backlog/enunciado/para}.
      }
    }
  },
  /deck/card/add 2 keys = {esfuerzo}{personas}{dias},
  format/.is choice,
  make id/.code = {H\refstepcounter{reqId}\arabic{reqId}\label{\pgfkeysvalueof{/ings/backlog/identificador}}},
  %
  % Nulo
  %
  format/none/.style = { /deck/card/.code = {}, },
  %
  % Lista
  %
  format/list/.style = {%
    /deck/card/.code = {%
      \pgfkeysalso{##1} % Visible desde toda la tabla
      \begin{tabular}{rp{11cm}}
        \textbf{Identificador}      & \pgfkeysalso{make id}                                     \\
        \textbf{Enunciado}          & \pgfkeysvalueof{/ings/backlog/enunciado}                  %%
          {\itshape\color{darkgray}   \pgfkeysvalueof{/ings/backlog/comentarios}}               \\
        \textbf{Valor cliente}      & \pgfkeysvalueof{/ings/backlog/valor cliente}              \\
        \textbf{Esfuerzo}           & \Plural{/ings/backlog/esfuerzo/personas}{persona} $\cdot$
                                      \Plural{/ings/backlog/esfuerzo/dias}{dia}                 \\
        \textbf{Iteracion}          & \pgfkeysvalueof{/ings/backlog/iteracion}                  \\
        \textbf{Prioridad}          & \pgfkeysvalueof{/ings/backlog/prioridad}                  \\
      \end{tabular}
      \vspace{0.33cm}\newline
      \rule{\textwidth}{1pt}\vspace{0.33cm}
      \newline
    }
  },
  %
  % TikZ
  %
  format/tikz/.style = {%
    /deck/card/.code = {%
      \pgfkeysalso{##1}%
      \def\CardWidth{0.5\textwidth}%
      \begin{tikzpicture}[%[anchor = base, baseline] % =(X.base)
        ystep=-1cm,
        score/.style = {%
          draw=####1, fill=####1!10, thick, inner sep=0pt, outer sep=0pt, minimum height=7mm, minimum width=10mm,
        },
        every label/.style = { font = \scshape },
        block/.style = {%
          draw=black, anchor=south,text width=\CardWidth-18pt, inner sep = 9pt,
        }]

        \node (titulo)
          [anchor=north, block, align=center, fill=black!7]
          {\scshape Historia \pgfkeysalso{make id}};

        \node (descripcion)
          [block, align=justify, below=0pt of titulo, text depth=14em]
          {\pgfkeysvalueof{/ings/backlog/enunciado} \newline
           {\itshape\color{darkgray}\pgfkeysvalueof{/ings/backlog/comentarios}}};

        \node (puntuacion) [above=30pt of descripcion.south] {};

        \node (valor)
          [score={yellow}, label={left:Valoración}, above left=0pt of puntuacion]
          {\pgfkeysvalueof{/ings/backlog/valor cliente}};

        \node (esfuerzo)
          [score={green}, label=left:Esfuerzo, below left=0pt of puntuacion]
          { \pgfkeysvalueof{/ings/backlog/esfuerzo/personas}
          / \pgfkeysvalueof{/ings/backlog/esfuerzo/dias}};

        \node (iteracion)
          [score={blue}, label=right:Iteración,, above right=0pt of puntuacion]
          {\pgfkeysvalueof{/ings/backlog/iteracion}};

        \node (prioridad)
          [score={red}, label=right:Prioridad,, below right=0pt of puntuacion]
          {\pgfkeysvalueof{/ings/backlog/prioridad}};

      \end{tikzpicture}\hskip0pt%
    }
  },
  %
  % Tabla
  %
  format/table/.style = {%
    /deck/.add code = {%
      \tabular{|r|p{9.2cm}|l|l|l|l|}
        \hline
        Id. & Enunciado y comentarios & \multicolumn{4}{c|}{Valoraciones}  \\ \cline{3-6}
            &                         & Val & Esf & Ite & Pri              \\
        \hline
    }{%
      &&&&&\\\hline\endtabular
    },
    /deck/card/.code = {%
      % Cada celda forma un grupo aislado, y no tenemos ninguna manera de definir
      % claves para toda la fila sin usar globales (que prefiero evitar).
      % Lo que hacemos es re-ejecutar las claves en cada celda.
        \pgfkeysalso{##1}\pgfkeysalso{make id}
      & \pgfkeysalso{##1}\pgfkeysvalueof{/ings/backlog/enunciado}
        { \color{darkgray} \itshape \pgfkeysvalueof{/ings/backlog/comentarios} }
      & \pgfkeysalso{##1}\pgfkeysvalueof{/ings/backlog/valor cliente}
      & \pgfkeysalso{##1}\pgfkeysvalueof{/ings/backlog/esfuerzo/personas}
                 $\cdot$ \pgfkeysvalueof{/ings/backlog/esfuerzo/dias}
      & \pgfkeysalso{##1}\pgfkeysvalueof{/ings/backlog/iteracion}
      & \pgfkeysalso{##1}\pgfkeysvalueof{/ings/backlog/prioridad} \\%
    },
  }
}

%% \ProductBacklog. Recibe un unico argumento, un conjunto de
%% claves de pgfkeys que tendra varias veces el comando `add element`
\NewDocumentCommand{\ProductBacklog}{O{}m}{%
  \begingroup
  \pgfqkeys{/ings/backlog}{%
    format=list,
    #1, % Extra opts.
    /deck = {#2},
  }
  \endgroup
}

\begin{document}

%% Formatos validos: none, list, tikz, table
\ProductBacklog[format=tikz]{%
{%
  identificador = Autenticacion,
  enunciado     = {%
    como = {director de proyecto, ingeniero o cliente},
    quiero = {autenticarme en el sistema},
    para = {acceder a mis opciones y mis proyectos},
  },
  valor cliente = 5,
  esfuerzo      = {2}{3},
  iteracion     = 1,
  prioridad     = 5,
  comentarios   = {Funcionalidad básica de la aplicación},
},{%
  identificador = CrearProyectos,
  enunciado     = {%
    como = {director de proyecto},
    quiero = {poder crear proyectos},
    para = {añadir nuevos proyectos},
  },
  valor cliente = 5,
  esfuerzo      = {1}{1},
  iteracion     = 1,
  prioridad     = 5,
  comentarios   = {Funcionalidad básica de la aplicación},
},{%
  identificador = ListarProyectos,
  enunciado     = {%
    como = {director de proyecto, ingeniero o cliente},
    quiero = {listar los proyectos en los que estoy involucrado},
    para = {poder acceder a ellos},
  },
  valor cliente = 4,
  esfuerzo      = {1}{1},
  iteracion     = 1,
  prioridad     = 4,
  comentarios   = {},
},{%
  identificador = ListarRequisitos,
  enunciado     = {%
    como = {director de proyecto o ingeniero},
    quiero = {acceder a los requisitos asociadas a un proyecto},
    para = {poder visualizarlos y comprobar si son correctos y completos},
  },
  valor cliente = 5,
  esfuerzo      = {3}{2},
  iteracion     = 1,
  prioridad     = 4,
  comentarios   = {},
},{%
  identificador = ListarSolicitudes,
  enunciado     = {%
    como = {director de proyecto o ingeniero},
    quiero = {acceder a las solicitudes asociadas a un proyecto},
    para = {poder visualizarlos y comprobar si son correctos y completos},
  },
  valor cliente = 5,
  esfuerzo      = {3}{2},
  iteracion     = 1,
  prioridad     = 4,
  comentarios   = {},
},{%
  identificador = CrearRequisitos,
  enunciado     = {%
    como = {director de proyecto o ingeniero},
    quiero = {poder añadir requisitos a mis proyectos siguiendo un formato estándar y desde la propia aplicación},
    para = {unificar todos los requisitos},
  },
  valor cliente = 5,
  esfuerzo      = {2}{2},
  iteracion     = 1,
  prioridad     = 5,
  comentarios   = {},
},{%
  identificador = ModificarRequisitos,
  enunciado     = {%
    como = {usuario},
    quiero = {modificar requisitos de mis proyectos},
    para = {completarlos o corregirlos en etapas posteriores a la creación},
  },
  valor cliente = 2,
  esfuerzo      = {2}{2},
  iteracion     = 2,
  prioridad     = 2,
  comentarios   = {},
},{%
  identificador = FiltrarRequisitos,
  enunciado     = {%
    como = {usuario},
    quiero = {consultar los requisitos de mis proyectos en función de ciertos parámetros},
    para = {poder encontrarlos más fácilmente},
  },
  valor cliente = 1,
  esfuerzo      = {1}{1},
  iteracion     = 2,
  prioridad     = 1,
  comentarios   = {},
},{%
  identificador = CrearSolicitudes,
  enunciado     = {%
    como = {usuario},
    quiero = {crear solicitudes una vez se haya hecho la primera etapa de deducción de requisitos},
    para = {poder debatir la modificación o inclusión de nuevos requisitos en el proyecto},
  },
  valor cliente = 3,
  esfuerzo      = {2}{2},
  iteracion     = 2,
  prioridad     = 3,
  comentarios   = {},
},{%
  identificador = CrearArtefactos,
  enunciado     = {%
    como = {ingeniero},
    quiero = {crear artefactos para poder añadir prototipos, diagramas de transición de estados, componentes, clases, planes de prueba, casos de prueba y otros},
    para = {el desarrollo de los requisitos},
  },
  valor cliente = 5,
  esfuerzo      = {4}{3},
  iteracion     = 1,
  prioridad     = 5,
  comentarios   = {},
},{%
  identificador = AsignarArtefactos,
  enunciado     = {%
    como = {ingeniero},
    quiero = {asociar artefactos creados a requisitos de un proyecto},
    para = {tenerlos agrupados en el requisito correspondiente},
},
  valor cliente = 4,
  esfuerzo      = {1}{1},
  iteracion     = 1,
  prioridad     = 4,
  comentarios   = {},
},{%
  identificador = ListarArtefactos,
  enunciado     = {%
    como = {director de proyecto o ingeniero},
    quiero = {listar los artefactos de un proyecto},
    para = {consultarlos agrupados por requisitos},
  },
  valor cliente = 3,
  esfuerzo      = {1}{1},
  iteracion     = 1,
  prioridad     = 3,
  comentarios   = {},
},{%
  identificador = CrearReporte,
  enunciado     = {%
    como = {director de proyecto},
    quiero = {crear un reporte básico con información de los requisitos del proyecto y su estado},
    para = {mostrárselo al cliente y ver si la planificación se estará llevando a cabo},
  },
  valor cliente = 5,
  esfuerzo      = {4}{3},
  iteracion     = 2,
  prioridad     = 4,
  comentarios   = {},
},{%
  identificador = AsignarHorarios,
  enunciado     = {%
    como = {director de proyecto},
    quiero = {asignar horarios a los ingenieros controlando el número de horas},
    para = {hacer un reparto de tareas sin sobreasignaciones de trabajo},
  },
  valor cliente = 3,
  esfuerzo      = {3}{3},
  iteracion     = 2,
  prioridad     = 2,
  comentarios   = {},
},{%
  identificador = Revisarsolicitudes,
  enunciado     = {%
    como = {jefe de proyecto},
    quiero = {poder aceptar o rechazar solicitudes de inclusión de requisitos},
    para = {incluir o rechazar las tareas solicitadas por el cliente.},
  },
  valor cliente = 3,
  esfuerzo      = {2}{3},
  iteracion     = 2,
  prioridad     = 1,
  comentarios   = {},
},{%
  identificador = CrearRequisitos2,
  enunciado     = {%
    como = {jefe de proyecto},
    quiero = {añadir requisitos a mis proyectos},
    para = {posteriormente asignar tareas a los ingenieros},
  },
  valor cliente = 3,
  esfuerzo      = {2}{2},
  iteracion     = 2,
  prioridad     = 3,
  comentarios   = {},
}} % /ProductBacklog

\end{document}