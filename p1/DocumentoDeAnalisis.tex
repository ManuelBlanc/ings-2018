\documentclass[a4paper, spanish]{article}

\usepackage[T1]{fontenc}
\usepackage[utf8]{inputenc}
\usepackage{babel, parskip}
\usepackage[colorlinks]{hyperref}

\title{Documento de Analisis}
\author{Autor McAutorson}

\begin{document}

\maketitle
\begin{abstract}
No por mucho madrugar amanece mas temprano.
No por mucho madrugar amanece mas temprano.
No por mucho madrugar amanece mas temprano.
No por mucho madrugar amanece mas temprano.
No por mucho madrugar amanece mas temprano.
No por mucho madrugar amanece mas temprano.
No por mucho madrugar amanece mas temprano.
\end{abstract}
\vspace{\fill}
\tableofcontents
%% Estas dos lineas son temporales
%\pagebreak
\let\oldsection\section\renewcommand\section{\clearpage\oldsection}

\section{Descripción del Sistema}
Parrafo introductorio del sistema que se va a desarrollar.

\subsection{Descripción y Motivación}
Aqui se describe un poco mas a fondo el sistema, y la motivacion
que guia el desarrollo.

\subsection{Objetivos del Sistema}
Que objetivos finales tiene el sistema.

\subsection{Usuarios del Sistema}
Que usuarios van a interactuar con el sistema.

\section{Definición del Sistema}
Parrafo introductorio a la seccion.

\subsection{Modelado de Casos de Uso}
Aqui iria un bonito diagrama UML de los casos de uso.

\subsection{Requisitos del sistema}
Para cada caso de uso, aqui se incluye un bloque que lo describe mas a fondo.
\newcommand{\CasoDeUso}[1]{Aun no implementado}
\CasoDeUso{%
	identificador          = {...},
	autor/modificacion     = {...},
	fecha                  = {...},
	actores involucrados   = {...},
	resumen                = {...},
	pre condiciones        = {...},
	post condiciones       = {...},
	% TODO: Revisar estos campos. No estan del todo claros en la plantilla.
	usuario                = {...},
	sistema                = {...},
	caminos alternativos   = {...},
	clases involucradas    = {...},
}

\section{Glosario}
\newcommand{\Glosario}[1]{Aun no implementado}
\Glosario{%
	Termino 1 = {Definicion 1},
	Termino 2 = {Definicion 2},
	Termino 3 = {Definicion 3},
}

\end{document}
