\documentclass[a4paper, 12pt, spanish]{memoria}

\usepackage{ings}

\begin{document}

% En los DM, cada miembro del equipo debe responder las siguientes preguntas:
% - ¿Qué hice ayer que ayudó al equipo de desarrollo a lograr el objetivo del Sprint?
% - ¿Qué haré hoy para ayudar al equipo de desarrollo a lograr el objetivo del Sprint?
% - ¿Veo algún impedimento que evite que el equipo de desarrollo o yo logremos el objetivo del Sprint?

\section*{Reuniones diarias}

\subsection*{Reunión del 2018-02-22}
Esta reunión se llevo a cabo en el laboratorio 12. El secretario fue Manuel.

\paragraph{Discusión general}
\begin{itemize}
\item Antes de ponerse a trabajar el equipo revisa el Product/Sprint Backlog.
\item Pablo realiza un reparto preliminar de las tareas, y empieza a crearlas en el tablero de GitHub.
\item Pablo pregunta si solo se ponen las historias del primer sprint, o todo. El equipo lo discute, y cree mejor solo poner las tareas del primer sprint por claridad.
\item David pregunta si los diagramas de clase son independientes. El equipo lo discute y llega a la conclusión que tiene mas sentido hacer un único diagrama de clases para toda la práctica, ya que hay ciertos componentes difíciles de aislar. Si se hiciesen diagramas de clase para cada caso de uso, habría mucha redundancia.
\item Empezamos a pensar el diagrama de clases.
\item Pablo llama la atención de que ser Jefe de Proyecto es una relación mas que una clase.
\item Cada proyecto puede tener un jefe de proyecto.
\item David formula el sistema en formato Entidad-Relación.
\item Ferpi comenta que un jefe puede ser de toda la empresa o de solo un proyecto.
\item David dice que solo pondría Cliente o Ingeniero, teniendo la relación ``jefe de''.
\item Alejandro nos recomiendo crear 3 clases Cliente, Ingeniero y Jefe para simplificar.
\item Jari hace el diagrama de casos de uso.
\item Jari pregunta si hay que poner al admin.
\item Ferpi escribe el product backlog.
\end{itemize}

\subsection*{Reunión del 2018-03-01}
Esta reunión se llevo a cabo en el laboratorio 12. El secretario fue Manuel.

\paragraph{¿Qué ha hecho el equipo?}
\begin{itemize}
    \item Ferpi ha escrito un borrador de los requisitos.
    \item Jari ha hecho el diagrama de casos de uso.
    \item Nevado ha ayudado con la coordinación del equipo, asegurándose de que las distintas partes del documento sean coherentes entre si.
    \item Pablo ha hecho diagrama de clases.
    \item Manu ha mecanografiado los requisitos y ha esbozado las secciones del documento.
\end{itemize}

\paragraph{¿Qué vamos a hacer?}
\begin{itemize}
  \item Nevado va a hacer los diagramas de secuencia.
  \item Pablo y Jari van a hacer maquetas de las páginas principales de la aplicacion.
  \item Ferpi va a ayudar a redactar los documentos.
  \item Manu va a centrarse en la parte relacionada al \LaTeX{} y en la redacción.
\end{itemize}
\paragraph{¿Qué impedimentos hay?}
\begin{itemize}
  \item Examen de EDP a la vuelta de la esquina.
  \item Hubo que revisar el trabajo anterior, ya que los contenidos del diagrama de clases y el de secuencia están fuertemente acoplados entre sí.
\end{itemize}


\subsection*{Reunión del 2018-03-08}
Esta reunión se llevo a cabo en el laboratorio 12. El secretario fue Manuel. \newline
\textbf{NOTA}: Este día hubo huelga.

\paragraph{¿Qué ha hecho el equipo?}
\begin{itemize}
  \item Jari ha terminado el diagrama de casos de uso.
  \item Ferpi, Pablo y David han hecho pequeñas correcciones en varios diagramas.
  \item Pablo y Jari han terminado las maquetas.
  \item Manu ha escribio los requisitos.
\end{itemize}

\paragraph{Que vamos a hacer?}
\begin{itemize}
        \item Manu va a revisar el \LaTeX{} de los documentos.
        \item Jari va a revisar los requisitos.
        \item Nevado va a revisar las maquetas.
        \item Pablo va a revisar los diagramas de secuencia. 
        \item Ferpi va a revisar el diagrama de clases y de casos de uso.
\end{itemize}

\end{document}
