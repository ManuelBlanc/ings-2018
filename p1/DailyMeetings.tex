\documentclass[a4paper, 12pt, spanish]{memoria}

\usepackage{ings}

\newcommand{\DailyMeeting}[1]{Esto aun no esta implementado}

\begin{document}

\section*{Reuniones diarias}

\DailyMeeting{
fecha = {2018-02-22},
lugar = {Aula 12},
acta = {%
Manu es el secretario de esta sesion.

David pregunta si los diagramas de clase son independientes.

Antes de ponerse a trabajar el equipo revisa el Product/Sprint Backlog.
Pablo reparte las tareas.

Pablo pregunta si solo se ponen las historias del primer sprint, o todo.

El equipo decide
Hace falta hacer un diagrama de uso para cada historia de usuario?
Hace falta los requisistos para la aplicacion en general, o por historia

En los diagramas de clase como se hacian relaciones de N a N.
EL euqipo revisa la documetnacion de UML.


Empezamos a pensar el diagrama de clases.
Pablo llama la atencion a que un Jefe de Proyecto no es una clase.
Cada proyecto puede tener un jefe de proyecto.
David lo formula en formato Entidad-Relacion.
Ferpi indica que puede haber dos tipos de JdP:
	- Como posicion en la empresa
	- El JdP de un proyecto
Jdp es un ingeniero especial

David dice que solo pondria Cliente o Ingeniero

Por simplificar Catalina nos dice crear las 3 clases: Cliente, Ingeniero y JdP.

Jari pregunta si alguien sabe UML.
Pregunta si hay que poner al admin.

},
% Cada miembro del equipo debe responder las siguientes preguntas:
% - ¿Qué hice ayer que ayudó al equipo de desarrollo a lograr el objetivo del Sprint?
% - ¿Qué haré hoy para ayudar al equipo de desarrollo a lograr el objetivo del Sprint?
% - ¿Veo algún impedimento que evite que el equipo de desarrollo o yo logremos el objetivo del Sprint?
} % /DailyMeeting

\end{document}