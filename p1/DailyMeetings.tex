\documentclass[a4paper, 12pt, spanish]{memoria}

\usepackage{ings}

\begin{document}

% En los DM, cada miembro del equipo debe responder las siguientes preguntas:
% - ¿Qué hice ayer que ayudó al equipo de desarrollo a lograr el objetivo del Sprint?
% - ¿Qué haré hoy para ayudar al equipo de desarrollo a lograr el objetivo del Sprint?
% - ¿Veo algún impedimento que evite que el equipo de desarrollo o yo logremos el objetivo del Sprint?

\section*{Reuniones diarias}

\subsection*{Reunión del 2018-02-22}
Esta reunión se llevo a cabo en el laboratorio 12. El secretario fue Manuel.

\paragraph{Discusión general}
\begin{itemize}
\item David pregunta si los diagramas de clase son independientes.
\item Pablo realiza un reparto preliminar de las tareas.
\item Pablo pregunta si solo se ponen las historias del primer sprint, o todo.
\item Antes de ponerse a trabajar el equipo revisa el Product/Sprint Backlog.
\item El equipo discute las siguientes cuestiones:
  \begin{itemize}
  \item Hace falta hacer un diagrama de uso para cada historia de usuario?
  \item Hace falta los requisitos para la aplicación en general, o por historia
  \end{itemize}
\item Empezamos a pensar el diagrama de clases.
\item Pablo llama la atención de que ser Jefe de Proyecto es una relación mas que una clase.
\item Cada proyecto puede tener un jefe de proyecto.
\item David formula el sistema en formato Entidad-Relación.
\item Ferpi comenta que un jefe puede ser de toda la empresa o de solo un proyecto.
\item David dice que solo pondría Cliente o Ingeniero, teniendo la relación ``jefe de''.
\item Catalina nos recomiendo crear 3 clases Cliente, Ingeniero y Jefe para simplificar.
\item Jari hace el diagrama de casos de uso.
\item Jari pregunta si hay que poner al admin.
\item Ferpi escribe el product backlog.
\end{itemize}

\subsection*{Reunión del 2018-03-01}
Esta reunión se llevo a cabo en el laboratorio 12. El secretario fue Manuel.

\paragraph{¿Qué ha hecho el equipo?}
\begin{itemize}
    \item Ferpi los requisitos
    \item Jari diagrama de uso
    \item Nevado resuelto dependencias
    \item Pablo diagrama de clases
    \item Manu ha hecho requisito y la descripción
\end{itemize}

\paragraph{¿Qué vamos a hacer?}
\begin{itemize}
        \item Manu hace \LaTeX.
        \item Jari hace maquetas.
        \item Nevado hace secuencia.
        \item Pablo hace maquetas.
        \item Ferpi va a redactar.
\end{itemize}
\paragraph{¿Qué impedimentos hay?}
\begin{itemize}
  \item Examen de EDP a la vuelta de la esquina.
  \item Resulta que la dependencia de tareas era al revés: \newline
    Hay que hacer primero el diagrama de clases y \textit{después} el de secuencia.
\end{itemize}


\subsection*{Reunión del 2018-03-08}
Esta reunión se llevo a cabo en el laboratorio 12. El secretario fue Manuel.

\paragraph{¿Qué ha hecho el equipo?}
\begin{itemize}
  \item Jari ha terminado el diagrama de casos de uso.
  \item Pablo ha terminado diagrama de clases.
  \item Manu ha hecho requisitos y la descripción.
  \item Ferpi ha revisado los requisitos.
  \item Jari y Pablo hicieron las maquetas.
\end{itemize}

\paragraph{Que vamos a hacer?}
\begin{itemize}
        \item Manu va a revisar el \LaTeX{} de los documentos.
        \item Jari va a revisar los requisitos.
        \item Nevado va a revisar las maquetas.
        \item Pablo va a revisar los diagramas de secuencia. 
        \item Ferpi va a revisar el diagrama de clases y de casos de uso.
\end{itemize}

\end{document}
