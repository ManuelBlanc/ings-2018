\documentclass[a4paper, 12pt, spanish]{memoria}

\usepackage{ings}

\begin{document}

\section*{Sprint Review}

\subsection*{Reunión del 2018-08-12}
Esta reunión se llevo a cabo delante del laboratorio 16. El secretario fue Manuel.

\paragraph{Evaluación del Sprint}
El equipo se dispone a conocer y evaluar el progreso del proyecto al final de este primer sprint.

Manuel presenta las metas de este sprint: se pretendía documentar en el documento de análisis y el de diseño los siguientes requisitos del proyecto:
\begin{itemize}
	\item Crear un proyecto nuevo (CrearProyecto).
	\item Añadir un requisito a un proyecto (CrearRequisito).
	\item Enviar una solicitud para incorporar un nuevo requisito (CrearSolicitud).
	\item Subir un artefacto entregable (CrearArtefacto).
\end{itemize}
Se han cumplido estos objetivos (que coincidían con los objetivos de la practica),
como se puede comprobar en el Sprint Backlog completado.

% presente al Product Owner y al resto de stakeholders (clientes y usuarios principalmente) las funcionalidades o requisitos que se han completado durante el sprint. 
Fernando, el Product Owner del grupo ha dado el visto bueno y esta satisfecho con el trabajo del equipo. Entre todos comprobamos que 

% El sprint review comienza con un miembro del equipo presentando las metas del sprint, el product backlog comprometido y el product backlog completado.
%Subiremos el sprint backlog inicial y el final para comparar

% Todos los miembros del equipo pueden comentar los aspectos que fueron o no satisfactorios durante el sprint.
\paragraph{Comentarios sobre el Sprint}
\begin{itemize}
\item David advierte mantenerse alejado de ArgoUML. Literalmente no tiene undo (véase \url{http://argouml.tigris.org/issues/show_bug.cgi?id=1834})
\item Pablo advierte de tener cuidado con el modo trial de algunas paginas de UML, que no almacenan los proyectos.
%\item Aprendizaje de uso de GIMP.
\item David habla de que planificamos con una linealidad que luego no fue cierta en la practica.
\item Manu señala que hubo dependencias cíclicas que hubo que resolver, por ejemplo entre el diagrama de clases, los requisitos y el diagrama de flujo.
%\item Pablo dice que mencionamos la planificación
\end{itemize}

% El acta a entregar con los resultados de esta reunión debe incluir, al menos, los requisitos que se han satisfecho, los que no y los que quedan pendientes para el próximo sprint.
%Satisfecho: los 4 planificados
%Los que no, los que quedan
%Los que quedan 


\end{document}
